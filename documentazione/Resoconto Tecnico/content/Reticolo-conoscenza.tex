\section{Costruzione del Reticolo della Conoscenza}
\label{Costruzione del Reticolo della Conoscenza}
Dal 26/07  al 28/07 ho iniziato la Costruzione del Reticolo della Conoscenza. Per realizzarlo ho utilizzato un'applicativo sviluppato durante un precedente stage dell'azienda, che fa largo uso della libreria d3.js; effettuando elaborazione di dati presentandoli in \textit{Cluster Based} o \textit{Force Based}, a discrezione delle esigenze dell'utente.\\
L'attivit\`a di creazione, documentazione e test del Reticolo \`e durata fino al termine dello stage.

\subsection{Descrizione del sistema}
\label{Descrizione del sistema}
L'applicazione accetta in import file di estensione CSV. Ogni colonna di quest'ultimo viene interpretata dal sistema come un parametro da elaborare; per questo la prima riga del file deve essere o preceduta da una dichiarazione di variabili, tante quante sono le colonne da parametrizzare, oppure \`e questa prima riga che viene interpretata come una dichiarazione e conseguentemente non rappresentata all'interno del Reticolo.\\
Nel sistema possono venire settati i seguenti aspetti:
\begin{itemize}
\item La \textit{tipologie} di Reticolo:
\begin{itemize}
\item Cluster Based: raggruppa un insieme di oggetti in modo tale che gli tutti gli elementi contenuti nel medesimo cluster sono pi\`u simili l'uno all'altro rispetto a quelli contenuti in altri gruppi;
\item Force Based: in base alla forza di ogni nodo viene rappresentata come unica regione compatta le istanze appartenenti alla medesima classe in cui vengono visivamente identificati i percorsi di differenziazione. Nel layout le celle differenzianti sono poste in prossimit\`a della classe pi\`u fortemente correlata.
\end{itemize}
\item \textit{Normalizzazione} dei dati in input:
\begin{itemize}
\item No: non viene applicata alcuna tecnica di normalizzazione dei dati;
\item MinMax: i dati vengono ridimensionati su un intervallo specifico (min, max), tuttavia tale tecnica non \`e in grado di gestire i valori anomali;
\item Gaussian: o normale in cui i dati vengono normalizzati in una curva in cui i valori della stessa grandezza sono soggetti ad approssimazione;
\item Interquartile: si occupa di standardizzare i dati in modo da quantificare l'estensione del 50\% della distribuzione del carattere che si trovano attorno alla mediana;
\end{itemize}
\item Tipologia di \textit{distanza} applicabili ai punti:
\begin{itemize}
\item Euclidea: tiene conto della distanza tra i punti;
\item Camberra: tiene conto della distanza tra le coppie in uno spazio vettoriale;
\item Pearson: distanza di correlazione che misura il grado di correlazione tra due punti. Valuta la covarianza tra due variabili in rapporto al prodotto della deviazione standard. Non \`e vantaggiosa su dati semplici.
\end{itemize}
\item \textit{Metodo} di associazione dei punti:
\begin{itemize}
\item Single: "vicino al prossimo", la distanza fra i gruppi \`e posta al pari della pi\`u piccola delle distanze calcolabili a due a due tra tutti gli elementi del gruppo. Accentua tutte le somiglianze tra i gruppi a discapito  della loro differenziazione netta.
\item Complete: "vicino pi\`u lontano", viene considerata la maggiore tra le distanze calcolate a due a due tra gli elementi di due gruppi. Privilegia la differenziazione tra i gruppi a discapito dell'omogeneit\`a degli elementi in essi contenuti. In questo caso i punti vengono rappresentati come meno compatti e diluiti.
\item Average: viene considerata come distanza fra due gruppi la media fra tutte le distanze calcolate a due a due tra gli elementi dei due gruppi. I risultati ottenibili sono i pi\`u attendibili (essendo basato sulla media delle distanze), i gruppi risultano pi\`u omogenei e differenziati tra di loro.
\end{itemize}
\end{itemize}
\noindent
Un'ulteriore funzionalit\`a permette all'utente di decidere se si desidera procedere con una rappresentazione del Reticolo manuale o automatica progressiva dei dati.

\subsubsection{Configurazione}
\label{configurazione}
Analizzando l'applicativo in base al carattere dei dati in ingresso e alle aspettative sull'output del modello, ho riscontrato che la configurazione necessaria per la formazione del Reticolo della Conoscenza \`e vincolata alla dischiarazione delle seguenti propriet\`a:
\begin{itemize}
\item \textit{Redistance}: No;
\item  \textit{Normalize}: No;
\item \textit{Distance-Type}: Euclidea;
\item \textit{Method}: Single.
\end{itemize}
\noindent
\begin{figure}[H]
\centering
	\includegraphics[width=1\linewidth]{./image/img-configurazione_reticolo.png}
	\caption{Configurazione usata nel sistema per generare il Reticolo della Conoscenza.}
	\label{Configurazione usata nel sistema per generare il Reticolo della Conoscenza.}
\end{figure}
\noindent

\subsection{Creazione dei file CSV}
\label{Creazione dei file CSV}
Come gi\`a accentato all'interno della sezione §{Descrizione del sistema} prima di procedere alla creazione del Reticolo ho dovuto preparare i dati di previsione prima di poterli dare in pasto al sistema. Questo \`e stato reso pi\`u agevole grazie alla creazione, da parte mia, di un metodo che ha il compito, una volta messa in funzione la Rete neurale oggetto di studio (di prova o del database), di calcolare:
\begin{enumerate}
\item Le previsioni ottenibili da un vettore di previsione settato a 1 o -1 per ogni singolo elemento;
\item Sui dati del punto (1) un vettore delle differenze dove viene calcolato il delta in rapporto al vettore di standard \footnote{Vettore tutto a zero}.
\end{enumerate}
\noindent
Ho fatto in modo che il vettore delle differenze venga stampato su console del browser \footnote{unica alternativa facendo uso di solo codice javascript}, in modo che ne basti prelevare il contenuto e inserirlo su un file CSV. Ogni elemento per poter funzionare all'interno dell'applicativo deve essere separato da un ; e ogni riga di previsione deve essere preceduta dal codice della domande in modo da rendere pi\`u agevole l'interpretazione del Reticolo. \`E a discrezione dell'utente l'inserimento di un'ulteriore riga di dichiarazione dei parametri.

\subsection{Creazione del Reticolo della Conoscenza per sui dati di Prova}
\label{Creazione del Reticolo della Conoscenza per sui dati di Prova}
\noindent
\begin{figure}[H]
\centering
	\includegraphics[width=1\linewidth]{./image/fileCSV_rete-prova.png}
	\caption{file CSV generato per la creazione del Reticolo della Conoscenza sui dati di prova.}
	\label{file CSV generato per la creazione del Reticolo della Conoscenza sui dati di prova.}
\end{figure}
\noindent
\textit{Di seguito sono riportate le sequenze di creazione del Reticolo della Conoscenza per i dati di prova.}
\noindent
\begin{figure}[H]
\centering
	\includegraphics[width=1\linewidth]{./image/reticoloCorretto1.png}
	\caption{Reticolo della Conoscenza con rappresentazione di tipo Cluster Based - 1.}
	\label{Reticolo della Conoscenza con rappresentazione di tipo Cluster Based - 1.}
\end{figure}
\noindent
\begin{figure}[H]
\centering
	\includegraphics[width=1\linewidth]{./image/reticoloCorretto2.png}
	\caption{Reticolo della Conoscenza con rappresentazione di tipo Cluster Based - 2.}
	\label{Reticolo della Conoscenza con rappresentazione di tipo Cluster Based - 2.}
\end{figure}
\noindent
\begin{figure}[H]
\centering
	\includegraphics[width=1\linewidth]{./image/reticoloCorretto3.png}
	\caption{Reticolo della Conoscenza con rappresentazione di tipo Forced Based - 1.}
	\label{Reticolo della Conoscenza con rappresentazione di tipo Forced Based - 1.}
\end{figure}
\noindent
\begin{figure}[H]
\centering
	\includegraphics[width=1\linewidth]{./image/reticoloCorretto4.png}
	\caption{Reticolo della Conoscenza con rappresentazione di tipo Forced Based - 2.}
	\label{Reticolo della Conoscenza con rappresentazione di tipo Forced Based - 2.}
\end{figure}
\noindent
\begin{figure}[H]
\centering
	\includegraphics[width=1\linewidth]{./image/reticoloCorretto5.png}
	\caption{Reticolo della Conoscenza con rappresentazione di tipo Forced Based - 3.}
	\label{Reticolo della Conoscenza con rappresentazione di tipo Forced Based - 3.}
\end{figure}
\noindent
\textit{Appare evidente come il Reticolo generato rispetta quando definito dalla figura \ref{Grafo rappresentante le relazioni esistenti tra il set di domande di prova.}.\\
Tuttavia effettuando delle ulteriori prove con file dati differenti; ma provenienti dalla medesima Rete neurale ho riscontrato situazioni contrastanti. L'errore osservato nei casi non corretti riguarda le coppie di domande 1, 4 e 3, 6. Un esempio del fenomeno \`e illustrato di seguito.}
\begin{figure}[H]
\centering
	\includegraphics[width=1\linewidth]{./image/reticoloNonCorretto1.png}
	\caption{Reticolo della Conoscenza con rappresentazione di tipo Forced Based - 3.}
	\label{Reticolo della Conoscenza con rappresentazione di tipo Forced Based - 3.}
\end{figure}
\noindent
\begin{figure}[H]
\centering
	\includegraphics[width=1\linewidth]{./image/reticoloNonCorretto2.png}
	\caption{Reticolo della Conoscenza con rappresentazione di tipo Cluster Based - 1.}
	\label{Reticolo della Conoscenza con rappresentazione di tipo Cluster Based - 1.}
\end{figure}
\noindent
\begin{figure}[H]
\centering
	\includegraphics[width=1\linewidth]{./image/reticoloNonCorretto3.png}
	\caption{Reticolo della Conoscenza con rappresentazione di tipo Cluster Based - 2.}
	\label{Reticolo della Conoscenza con rappresentazione di tipo Cluster Based - 2.}
\end{figure}
\noindent
\begin{figure}[H]
\centering
	\includegraphics[width=1\linewidth]{./image/reticoloNonCorretto4.png}
	\caption{Reticolo della Conoscenza con rappresentazione di tipo Cluster Based - 3.}
	\label{Reticolo della Conoscenza con rappresentazione di tipo Cluster Based - 3.}
\end{figure}
\noindent
\begin{figure}[H]
\centering
	\includegraphics[width=1\linewidth]{./image/reticoloNonCorretto5.png}
	\caption{Reticolo della Conoscenza con rappresentazione di tipo Forced Based - 1.}
	\label{Reticolo della Conoscenza con rappresentazione di tipo Forced Based - 1.}
\end{figure}
\noindent
\begin{figure}[H]
\centering
	\includegraphics[width=1\linewidth]{./image/reticoloNonCorretto6.png}
	\caption{Reticolo della Conoscenza con rappresentazione di tipo Forced Based - 2.}
	\label{Reticolo della Conoscenza con rappresentazione di tipo Forced Based - 2.}
\end{figure}
\noindent
\begin{figure}[H]
\centering
	\includegraphics[width=1\linewidth]{./image/reticoloNonCorretto7.png}
	\caption{Reticolo della Conoscenza con rappresentazione di tipo Forced Based - 3.}
	\label{Reticolo della Conoscenza con rappresentazione di tipo Forced Based - 3.}
\end{figure}
\noindent
\begin{figure}[H]
\centering
	\includegraphics[width=1\linewidth]{./image/reticoloNonCorretto8.png}
	\caption{Reticolo della Conoscenza con rappresentazione di tipo Forced Based - 4.}
	\label{Reticolo della Conoscenza con rappresentazione di tipo Forced Based - 4.}
\end{figure}
\noindent
La spiegazione del Reticolo malformato \`e da ricondurre all'applicativo usato per la sua rappresentazione. I punti vengono raggruppati, come spiegato nella sezione § \ref{Configurazione.} assieme usando la distanza euclidea, ci\`o implica che per effettuare i raggruppamenti viene calcolata la distanza che intercorre fra i punti dati in pasto al sistema. Fra due punti viene calcolata la distanza bidimensionale (TODO:da verificare) ovvero la radice quadrata delle differenze  nei due assi, sommate tra di loro. Nel momento in cui il gruppo viene creato il sistema procede all'individuazione del prossimo gruppo, effettuata mediante l'uso del metodo Complete. Significa che vengono presi tutti i punti in relazione stretta con i punti del gruppo in esame e scelto il pi\`u vicino; escludendo perci\`o da qualsiasi altra considerazione presenta e futura gli ulteriori punti coinvolti, se non solo quando chiamati in causa da altri punti.\\
\noindent
In fase di test mi sono anche occupata di capire come si comporta la Rete neurale, relativa ai dati di prova, considerando esclusivamente la possibilit\`a che un candidato o risponda correttamente ad una domanda (1) o la sbagli (-1). I risultati ottenuti sono i seguenti:
\begin{figure}[H]
\centering
	\includegraphics[width=1\linewidth]{./image/RetediProva_generatorinputpuro.png}
	\caption{Risultati della Rete di test con un set di dati puro (esclusivamente 1 e -1).}
	\label{Risultati della Rete di test con un set di dati puro (esclusivamente 1 e -1).}
	\end{figure}
	\noindent
Come si pu\`o vedere dalla figura \ref{Risultati della Rete di test con un set di dati puro (esclusivamente 1 e -1).} le previsioni ottenute per le coppie di domande (1;4), (3;6) e (2;5) sono identiche, ad eccezioni di alcune variazioni impercettibili da ricondurre ad oscillazioni di Rete e che non hanno alcun impatto sui risultati da raggiungere. Chiedendo una previsione sul valore 0 e avendo allenato la Rete esclusivamente con valori -1 e 1 mi aspetto che si venga effettuata la media, ed \`e quello che sembra venga fatto per tutte le domande coinvolte.
Il fenomeno sorprendente, che mette in luce l'importanza di avere un valore 0 di risposta (non data) nel trainset, \`e mostrato nelle figure sotto.

\begin{figure}[H]
\centering
	\includegraphics[width=1\linewidth]{./image/RetediProva_generatorinputpuro.png}
	\caption{Risultati della Rete di test con un set di dati puro (esclusivamente 1 e -1), vettore previsione [-1, 0, 0, 0, 0, 0].}
	\label{Risultati della Rete di test con un set di dati puro (esclusivamente 1 e -1), vettore previsione [-1, 0, 0, 0, 0, 0].}
\end{figure}
\noindent

\begin{figure}[H]
\centering
	\includegraphics[width=1\linewidth]{./image/RetediProva_generatorinputpuro.png}
	\caption{Risultati della Rete di test con un set di dati puro (esclusivamente 1 e -1), vettore previsione [0, 0, 0, -1, 0, 0].}
	\label{Risultati della Rete di test con un set di dati puro (esclusivamente 1 e -1), vettore previsione [0, 0, 0, -1, 0, 0].}
\end{figure}
\noindent

Non avere un valore a 0 porta le le previsioni imposte alle singole domande, valutate in coppia, a mostrare come unica relazione stretta esistente la coppia coinvolta, scollegando ogni relazione con le altre coppie. Come si vede dall'esempio sopra, le domande 2 e 4 vengono colorate di verdino nella previsione della domanda 1 e di rosso nella domanda 4. Questo ha un impatto sulla previsione visiva della Rete, dal quale risulta impossibile la previsione di domande figlie e genitori.\\
La matrice correlazione ottenuta, tuttavia dal modello della PCA, risulta perfettamente allineata con le aspettative delle coppie di domande e delle prevsioni, tuttavia questo \`e possibile solo perch\`e i dati e le previsioni vengono valutati nella loro generalit\`a e non tra coppia di domande. (TODO: manca SPIEGAZIONE 1 RETICOLO)

\subsubsection{Osservazioni}
\subsubsection{Osservazioni Reticolo dati di prova}
In conclusione il Reticolo della Conoscenza anche se non sempre perfettamente coerente con le aspettative; nella maggioranza dei casi ricalca fedelmente quanto evidenziato nella figura \ref{Grafo rappresentante le relazioni esistenti tra il set di domande di prova.} nella sezione § \ref{Test effettuati}; o comunque le deviazioni risultano minimali e motivate, come appena spiegato sopra. \\
Quando il Reticolo non \`e soggetto a deviazioni gli accoppiamenti tra i punti ricalcano quanto dichiarato dalla matrice correlazione ottenuta dal modello generato dall'uso della PCA, e visibile dall'immagine \ref{CSV generato a partire dalla matrice correlazione del trainset della rete di prova.}.
\`E obbligo precisare, tuttavia che i risultati della PCA non devono essere presi come assioma per le seguenti argomentazioni:
\begin{itemize} 
\item La Rete neurale, pu\`o cogliere oscillazioni che il modello matematico non \`e in grado e questo va ad invalidare i risultati ottenuti non solo dalla matrice correlazione ma anche dell'individuazione dei punti sulle prime due componenti;
\item La PCA non tiene conto del numero di volte in cui si verifica un evento, attribuisce il medesimo peso ad una domanda rivolta ad un utente 20 volte e 1 volta. Invece la Rete neurale mediante apprendimento non da un peso significativo a tali fenomeni. Tale osservazione porta a rendere inefficiente il modello risultante;
\item Lo scopo principale della PCA \`e quello di spostare gli assi di rappresentazione degli eventi in modo che vi sia una maggiore facilit\`a di comprensione dei dati; per intuire questo si basti pensare allo scopo della standardizzazione dei valori. Perci\`o la matrice correlazione risulta essere una stima da seguire che tiene conto compressione dei dati in esame e conseguentemente variabile dalle correlazioni reali.
\end{itemize}

\subsection{Creazione del Reticolo della Conoscenza sui dati delle domande nel database}
\label{Creazione del Reticolo della Conoscenza sui dati delle domande nel database}

\noindent
\begin{figure}[H]
\centering
	\includegraphics[width=1\linewidth]{./image/fileCSV_rete-db-10neuroni.png}
	\caption{Porzione di esempio file CSV generato per la creazione del Reticolo della Conoscenza sui dati del database.}
	\label{Porzione di esempio file CSV generato per la creazione del Reticolo della Conoscenza sui dati del database.}
\end{figure}
\noindent
La creazione del Reticolo della Conoscenza sui dati delle domande nel database ha richiesto la creazione di file CSV su un architettura a 2 layer della rete con 6, 8, 10, 12 neuroni ciascuno. Questo a causa della non conoscenza a priori del numero di cluster che comporranno il Reticolo, come invece accade per i dati di prova.\\
Tale differenziazione ha permesso durante la configurazione del sistema e la creazione del Reticolo che mi accorgersi dell'estrema variabilit\`a dei cluster e correlazioni a seconda dell'oscillazione del numero di neuroni per layers, come viene evidenziato dalle immagini seguenti.\\

TODO: INSERIMENTO DEI SCREEN DEL RETICOLO SUI CASI A 6, 8, 10, 12 NEURONI A CLUSTER E A FORCE BASED.

\subsection{Osservazioni}
\subsubsection{Osservazioni Reticolo dati del database}
La variazione dei Reticoli ottenuti dalle diverse architetture della Rete neurale indica come sia necessario approfondire la tematica e individuare una strategia unica che permetta di individuare in modo univoco l'architettura necessaria per ottenere un Reticolo della Conoscenza valido.

\subsection{Creazione del Reticolo prendendo in considerazione le frequenze alle domande}
\label{Creazione del Reticolo prendendo in considerazione le frequenze alle domande}
Ho ritenuto adeguato studiare il comportamento dei set di dati in base alla loro frequenza; in modo da poterne valutare le differenze con la Rete neurale e se vi fossero benefici o meno.\\
Come primo passo ho provveduto a calcolarmi per ogni domanda quale fosse la frequenza delle altre domande valutate in accordo e in disaccordo con la risposta data. Ho effettuato il procedimento sia nel caso di una domanda risposta correttamente che in modo sbagliato.


\noindent
\begin{figure}[H]
\centering
	\includegraphics[width=1\linewidth]{./image/res_frequenceMatrix_OSS.png}
	\caption{Frequenza delle domande dei dati di test.}
	\label{Frequenza delle domande dei dati di test.}
\end{figure}
L'immagine sopra mostra per ogni domanda valutata a 1 (primo set di dati) e a -1 (secondo set di dati) la frequenza che intercorre in tutte le domande con segno coerente o opposto a quello della domanda in esame.
Il set di dati viene generato randomicamente con guida \footnote{con impiego del Grafo della Conoscenza (mostrato in figura \ref{Grafo rappresentante le relazioni esistenti tra il set di domande di prova.})}.
Analizzando i risultati di frequenza osservati ho riscontrato i seguenti fenomeni:
\begin{itemize}
\item Considerando i valori per riga:
\begin{itemize}
\item per le domande in cui le risposte si presentano corrette i valori di frequenza correlata e opposta si presentano con la seguente struttura:
\begin{itemize}
\item la domanda 3  si presenta con una frequenza correlata che coinvolge le coppie (1,3) e (4,6);
\item la domanda 6 si presenta con una frequenza correlata che coinvolge le coppie (1,6) e (4,3).
\end{itemize}
\item per le domande in cui le risposte si presentano sbagliate i valori di frequenza correlata e opposta si presentano con la seguente struttura:
\begin{itemize}
\item la domanda 1 si presenta con una frequenza correlata che coinvolge le coppie (1 ~ FC:1, 3 ~ FC:1) e (4 ~ FC:0.55, 6 ~ FC:0.80);
\item la domanda 4 si presenta con una frequenza correlata che coinvolge le coppie (4 ~ FC:1, 3 ~ FC:1) e (1 ~ FC:0.54, 6 ~ FC:0.80);
\end{itemize}
\end{itemize}
\item Considerando i valori per colonna:
\begin{itemize}
\item la domanda 1 si correla con la domanda 4;
\item la domanda 3 si correla con la domanda 6, 
\item la domanda 2 con la domanda 5.
\end{itemize}
\end{itemize}
\noindent
Tali considerazioni non sono esclusive per il singolo set di dati; ma rimangono costanti per ogni set di dati generato (anche se di natura parzialmente randomica). Ho riscontrato per ciascuna variabile al massimo un oscillazione che si attesta mai superiore a 0.2. Percui tali risultati risultano stabili e attendibili.
\\\\
I valori di frequenza possono concorrere alla formazione del Reticolo della Conoscenza. Per renderlo possibile ho provveduto ha creare un file csv che contenesse per ognuna delle domande tutte le domande coinvolte valutate con frequenze positive correlate e opposte e le frequenze negative correlate e opposte. La tecnica permette la generazione, per ogni domanda, di un vettore contenete un quantit\`a di dati 4 volte la dimensione dell'input/output della Rete.
Il divario permette, positivamente, di poter costruire un Reticolo ove i dati per ogni nodo hanno una stabilit\`a concreta.\\
Tuttavia il Reticolo generato riscontra delle peculiarit\`a:
\begin{itemize}
\item la frequenza non effettua alcuna riduzione dimensionale provocando uno utilizzo di spazio pari al numero di domande/dati da analizzare. Ci\`o comporta che per grandi moli da dati lo spazio necessario cresce elevato al numero di elementi di input;
\item la frequenza funziona male sui valori anomali, come gi\`a spiegato nella sezione § \ref{Descrizione del sistema}
\end{itemize}
\noindent
Il divario tra i valori per riga e colonna sono legittimati dalla costruzione dei valori di input. Questi infatti vedono la presenta di valori 0, e non esclusivamente 0 e 1, il cui impatto  va a colpire ogni singola domanda, le cui conseguenze peggiori sono ottenute dalle domande figli e genitori che collidano tra loro dimenticando il grado di parentela. Tuttavia tale oscillazione ottiene un peso minimo durante il calcolo dei dati in quanto ui valori per colona e le righe 2 e 5 risultano sempre conformi.

\noindent
\begin{figure}[H]
\centering
	\includegraphics[width=1\linewidth]{./image/Reticolo_db-frequenza.png}
	\caption{Configurazione usata nel sistema per generare il Reticolo della Conoscenza con i dati di frequenza (Cluster Based) - 1.}
	\label{Configurazione usata nel sistema per generare il Reticolo della Conoscenza con i dati di frequenza (Cluster Based) - 1.}
\end{figure}
\noindent

\noindent
\begin{figure}[H]
\centering
	\includegraphics[width=1\linewidth]{./image/Reticolo_db-frequenza-2.png}
	\caption{Configurazione usata nel sistema per generare il Reticolo della Conoscenza con i dati di frequenza (Cluster Based) - 2.}
	\label{Configurazione usata nel sistema per generare il Reticolo della Conoscenza con i dati di frequenza (Cluster Based) - 2.}
\end{figure}
\noindent

\noindent
\begin{figure}[H]
\centering
	\includegraphics[width=1\linewidth]{./image/Reticolo_db-frequenza-3.png}
	\caption{Configurazione usata nel sistema per generare il Reticolo della Conoscenza con i dati di frequenza (Forced Based).}
	\label{Configurazione usata nel sistema per generare il Reticolo della Conoscenza con i dati di frequenza (Forced Based).}
\end{figure}
\noindent




