\section{Principal Component Analysis}
\label{PCA}

La \textbf{P}rincipal \textbf{C}omponent \textbf{A}nalysis\` (acronimo PCA) \`e una tecnica impiegata nell'ambito della statistica multivariata\footnote{parte della statistica in cui l'oggetto dell'analisi \`e almeno composta da due elementi} per semplificare i dati d'origine.\\
Lo scopo che tale tecnica persegue \`e lo studio della relazione esistenti tra i campioni d'interesse con la riduzione di un numero pi\`u o meno elevato di variabili. La riduzione dimensionale avviene tramite una trasformazione lineare della variabili che proietta quelle originarie in un nuovo sistema cartesiano in cui tutte le variabili vengono ordinate in maniera decrescente per ordine di varianza, successivamente la variabile con maggiore varianza viene proiettata sul primo asse, la seconda sul secondo e via via sempre cos\`i per tutte le variabili coinvolte. La PCA \`	e particolarmente utile quando la dimensionalit\`a dello spazio delle misure \`e elevata (molte colonne); ma i campioni si trovano in uno spazio di dimensioni significativamente ridotte. Indispensabile risulta essere la ricerca del numero di componenti principali significative, ovvero tutte le i variabili coinvolte a meno di quelle legate al rumore, che concorrono a comporre la \textit{dimensionalit\`a intrinseca}. Il "rumore" \`e sempre concentrato nelle ultime variabili , non includerle nell'analisi dei dati porta a dati pi\`u puliti, con un rapporto segnale/rumore pi\`u alto.
Il calcolo di quali sono le variabili pi\`u significative si ottiene come gi\`a detto sopra con la varianza. La prima componente principale spiega la massima percentuale della variabilit\`a presente nei dati rappresentabili in una singola dimensione, in poche parole la direzione lungo cui si registra la massima dispersione dei dati, tale percentuale pu\`o essere calcolata con la varianza. Quest'ultima porta il vantaggio di essere indipendente dal sistema di riferimento, conseguentemente una rotazione degli assi mantiene immutata la varianza totale all'interno dei dati.
